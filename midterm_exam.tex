\documentclass[10pt,legalpaper]{exam}
\printanswers
\usepackage[utf8]{inputenc}
\usepackage{graphicx}  % For including images
\graphicspath{{./img/}}
\DeclareGraphicsExtensions{.pdf,.jpeg,.jpg,.png}

\usepackage{multicol}
\usepackage{listings}  % For formatting code
\usepackage{enumitem}

% Define C++ syntax highlighting with color
\lstset{
    language=C++,
    basicstyle=\ttfamily\footnotesize, 		% Font style and size
    keywordstyle=\color{blue},         		% Color for keywords
    stringstyle=\color{red},           		% Color for strings
    commentstyle=\color{green!50!black}, 	% Color for comments
    numberstyle=\tiny\color{gray},    		% Line number color and size
    numbers=none,                      		% Line numbers on the left
    showstringspaces=false,            		% Don't display spaces in strings
    tabsize=4,                         		% Set tab width
    breaklines=true,                   		% Automatic line breaks
    breakatwhitespace=true,            		% Break lines at whitespace
    captionpos=b                       		% Caption at bottom
}


\usepackage{geometry}  % To adjust margins
\usepackage{xcolor} % For text color
\usepackage{ifthen}  % For conditional statements
\usepackage[utf8]{inputenc}  % Required for UTF-8 support
\usepackage{array}  % For advanced table formatting
\usepackage{booktabs}  % For better-looking tables (optional)
\usepackage{tcolorbox}

% Define a toggle: set to true to show answers, false to hide them
\newboolean{showanswers}
\setboolean{showanswers}{false}  % Change this to true to show correct answers

% Adjusting margins
\geometry{
    top=0.5in,     % Top margin
    bottom=1in,   % Bottom margin
    left=0.6in,     % Left margin
    right=0.6in,    % Right margin
}

% Title and general formatting
\title{\vspace{-1.7cm}MIDTERM EXAMINATION\\ IT 304 -- NETWORKING 2}
\author{Prepared by: Jaafar J. Omar}
\date{}

\begin{document}

\maketitle

% Student information
\noindent
Name:\underline{\hspace{8cm}} \hfill Date:\underline{\hspace{5cm}} \hfill Section:\underline{\hspace{8cm}}

\vspace{0.5cm}

\noindent
\textbf{Instructions:}

\noindent
Please read each question carefully before selecting/providing your answer. Choose the most accurate answer for each question.

\vspace{0.5cm}

% Change to a single column layout
\begin{questions}

	% Q1
	\question  The rules that govern network communications including the message format, message size, timing, and encapsulation, are known as network: \\
	\begin{oneparchoices}
		\choice signaling
		\CorrectChoice protocols
		\choice messaging
		\choice encoding
	\end{oneparchoices}

	\question Which protocol is responsible for guaranteeing reliable delivery? \\
	\begin{oneparchoices}
		\CorrectChoice TCP
		\choice Ethernet
		\choice HTTP
		\choice IP
	\end{oneparchoices}

	\question Which protocol is used by routers to forward messages? \\
	\begin{oneparchoices}
		\choice Ethernet
		\choice HTTP
		\choice TCP
		\CorrectChoice IP
	\end{oneparchoices}

	\question[2] Which two layers of the OSI Model maps directly as the single network access layer in the TCP/IP model? (Choose two) \\
	\begin{oneparchoices}
		\CorrectChoice data link
		\choice application
		\CorrectChoice physical
		\choice transport
		\choice presentation
		\choice network
	\end{oneparchoices}

	\question IP addressing occurs at what layer of the OSI Model? \\
	\begin{oneparchoices}
		\choice application
		\choice physical
		\choice transport
		\CorrectChoice network
	\end{oneparchoices}

	\question What networking term describes a particular set of rules at one layer that govern communication at that layer? \\
	\begin{oneparchoices}
		\choice duplex
		\choice encapsulation
		\choice error checking
		\CorrectChoice protocol
	\end{oneparchoices}

	\question Which of the following is NOT a criterion for choosing a network media?\\
	\begin{oneparchoices}
		\CorrectChoice The type of data that can be transmitted.
		\choice The maximum distance that the media can successfully carry a signal. \\
		\choice The environment in which the media is installed.
		\choice The cost of installing the media.
	\end{oneparchoices}

	\question What is the purpose of the OSI physical layer?
	\begin{choices}
		\choice controlling access to media.
		\choice transmitting bits across the local media.
		\choice performing error detection on received frames.
		\CorrectChoice exchanging frames between nodes over physical network media.
	\end{choices}

	\question Which statement is correct about network protocols?
	\begin{choices}
		\choice Network protocols define the type of hardware that is used and how it is mounted in racks.
		\CorrectChoice They define how messages are exchanged between the source and the destination.
		\choice They all function in the network access layer of TCP/IP.
		\choice They are only required for exchange of messages between devices on remote networks.
	\end{choices}

	\question What is the purpose of protocols in data communications?
	\begin{choices}
		\choice specifying the bandwidth of the channel or medium for each type of communication.
		\choice specifying the device operating systems that will support the communication.
		\CorrectChoice providing the rules required for a specific type of communication to occur.
		\choice dictating the content of the message sent during communication.
	\end{choices}

	\question The process of prepending protocol information with information from another protocol is called: \\
	\begin{oneparchoices}
		\choice encoding
		\choice framing
		\choice packetizing
		\CorrectChoice encapsulation
	\end{oneparchoices}

	\question When an Ethernet frame is sent out an interface, the destination MAC address indicates:
	\begin{choices}
		\choice The MAC address of the NIC card of a device, which is on this network or another network, that will receive the Ethernet frame.
		\choice The MAC address of the NIC card that sent the Ethernet frame.
		\CorrectChoice The MAC address of the device, which is on this network, that will receive the Ethernet frame.
		\choice The MAC address of the router.
	\end{choices}

	\question What is one function of a Layer 2 switch?
	\begin{choices}
		\choice forwards data based on logical addressing.
		\choice duplicates the electrical signal of each frame to every port.
		\choice learns the port assigned to a host by examining the destination MAC address.
		\CorrectChoice determines which interface is used to forward a frame based on the destination MAC address.
	\end{choices}

	\newpage

	\question Which type of address does a switch use to build the MAC address table? \\    \begin{oneparchoices}
		\choice destination MAC address
		\CorrectChoice source MAC address
		\choice destination IP address
		\choice source IP address
	\end{oneparchoices}

	\question Refer to the exhibit below. The exhibit shows a small switched network and the contents of the MAC address table of the switch. PC1 has sent a frame addressed to PC3. What will the switch do with the frame?
	\begin{table}[h]
		\centering
		\begin{tabular}{ c }
			\begin{minipage}{0.65\textwidth} % Use minipage to better control layout inside table
				\centering
				\includegraphics[width=0.6\textwidth]{./img/switched_network}
			\end{minipage} \\
		\end{tabular}
	\end{table}
	\begin{choices}
		\choice The switch will discard the frame.
		\choice The switch will forward the frame only to port 2.
		\CorrectChoice The switch will forward the frame to all ports except port 4.
		\choice The switch will forward the frame to all ports.
	\end{choices}

	\question How many octets exist in an IPv4 address? \\
	\begin{oneparchoices}
		\CorrectChoice 4
		\choice 8
		\choice 16
		\choice 32
	\end{oneparchoices}

	\question How large are IPv4 addresses?
	\begin{oneparchoices}
		\choice 8 bits
		\choice 16 bits
		\choice 32 bits
		\choice 64 bits
	\end{oneparchoices}

	\question What is the network number for an IPv4 address 172.16.34.10 with the subnet mask of 255.255.255.0? \\
	\begin{oneparchoices}
		\choice 10
		\choice 34.10
		\choice 172.16.0.0
		\CorrectChoice 172.16.34.0
	\end{oneparchoices}

	\question Host-A has the IPv4 address and subnet mask 10.5.4.100 255.255.255.0. What is the network address of Host-A? \\
	\begin{oneparchoices}
		\choice 10.5.4.100
		\CorrectChoice 10.5.4.0
		\choice 10.0.0.0
		\choice 10.5.0.0
	\end{oneparchoices}

	\question Host-B has the IPv4 address and subnet mask 172.16.4.100 255.255.0.0. What is the network address of Host-B? \\
	\begin{oneparchoices}
		\choice 172.16.4.100
		\choice 172.0.0.0
		\CorrectChoice 172.16.0.0
		\choice 172.16.4.0
	\end{oneparchoices}

	\question[2] Host-A has the IPv4 address and subnet mask 10.5.4.100 255.255.255.0. Which of the following IPv4 addresses would be on the same network as Host-A? (Choose all that apply) \\
	\begin{oneparchoices}
		\choice 10.0.0.98
		\choice 10.5.0.1
		\CorrectChoice 10.5.4.1
		\choice 10.5.100.4
		\CorrectChoice 10.5.4.99
	\end{oneparchoices}

	\question[2] Host-B has the IPv4 address and subnet mask 172.16.4.100 255.255.0.0. Which of the following IPv4 addresses would be on the same network as Host-B? (Choose all that apply) \\
	\begin{oneparchoices}
		\choice 172.17.4.99
		\choice 172.17.4.1
		\choice 172.18.4.1
		\CorrectChoice 172.16.4.99
		\CorrectChoice 172.16.0.1
	\end{oneparchoices}

	\question[2] Host-C has the IPv4 address and subnet mask 192.168.1.50 255.255.255.0. Which of the following IPv4 addresses would be on the same network as Host-C? (Choose all that apply) \\
	\begin{oneparchoices}
		\choice 192.168.2.1
		\choice 192.168.0.1
		\CorrectChoice 192.168.1.100
		\choice 192.168.0.100
		\CorrectChoice 192.168.1.1
	\end{oneparchoices}

	\question A host is transmitting a broadcast. Which host or hosts will receive it? \\
	\begin{oneparchoices}
		\CorrectChoice all hosts in the same network.
		\choice a specially defined group of hosts. \\
		\choice the closest neighbor on the same network.
		\choice all hosts on the internet.
	\end{oneparchoices}

	\question Which statement describes one purpose of the subnet mask setting for a host?
	\begin{choices}
		\choice It is used to describe the type of the subnet.
		\choice It is used to identify the default gateway.
		\CorrectChoice It is used to determine to which network the host is connected.
		\choice It is used to determine the maximum number of bits within one packet that can be placed on a particular network.
	\end{choices}

	\question Determine how many network bits and host bits are there in a Class C IPv4 address.\\
	\begin{oneparchoices}
		\choice Network bits: 8, Host bits: 24
		\CorrectChoice Network bits: 24, Host bits: 8 \\
		\choice Network bits: 16, Host bits: 16
		\choice Network bits: 32, Host bits: 0
	\end{oneparchoices}

	\question Determine how many network bits and host bits are there in a Class A IPv4 address. \\
	\begin{oneparchoices}
		\CorrectChoice Network bits: 8, Host bits: 24
		\choice Network bits: 16, Host bits: 16 \\
		\choice Network bits: 24, Host bits: 8
		\choice Network bits: 32, Host bits: 0
	\end{oneparchoices}

	\newpage

	\question[20] Given an IP address and Subnet Mask of \textbf{172.16.0.0 255.255.0.0}, answer the following question if the number of subnets created is 8. (Show your solution)
	\begin{parts}
		\part Determine how many bits you should borrow:
		\part Determine the subnet mask:
		\part Determine the usable hosts per subnet:
		\part Determine the network address and the broadcast address for each subnet.
		\begin{table}[h]
			\centering
			\begin{tabular}{l l l}
				\toprule
				           & \textbf{Network Address} & \textbf{Broadcast Address} \\ \midrule
				1st subnet &                          &                            \\ \hline
				2nd subnet &                          &                            \\ \hline
				3rd subnet &                          &                            \\ \hline
				4th subnet &                          &                            \\ \hline
				5th subnet &                          &                            \\ \hline
				6th subnet &                          &                            \\ \hline
				7th subnet &                          &                            \\ \hline
				8th subnet &                          &                            \\ \bottomrule
			\end{tabular}
		\end{table}
	\end{parts}

	\vspace{1cm}

	\begin{tcolorbox}[title=IP Address Choices]
		\begin{oneparchoices}
			\choice \textbf{10.0.0.5}        % Private A
			\choice \textbf{192.168.10.20}   % Private B
			\choice \textbf{172.16.5.8}      % Private C
			\choice \textbf{192.169.1.50}    % Public D
			\choice \textbf{172.32.15.45}    % Public E
			\choice \textbf{10.255.255.255}  % Private F
			\choice \textbf{172.31.255.255}  % Private G
			\choice \textbf{192.167.10.15}   % Public H
			\choice \textbf{11.0.0.5}        % Public I
			\choice \textbf{172.15.0.1}      % Public J
			\choice \textbf{8.8.8.8}         % Public K
			\choice \textbf{172.217.15.110}  % Public L
			\choice \textbf{198.51.100.45}   % Public M
			\choice \textbf{203.0.113.65}    % Public N
			\choice \textbf{131.107.255.255} % Public O 
			\choice \textbf{10.1.1.1}        % Private P
			\choice \textbf{172.16.0.10}     % Private Q
			\choice \textbf{192.168.50.5}    % Private R
			\choice \textbf{172.31.10.200}   % Private S 
			\choice \textbf{192.168.1.10}    % Private T
		\end{oneparchoices}
	\end{tcolorbox}

	\question[10] Refer to the IP addresses above to identify the \emph{private} IPv4 addresses:
	\question[10] Refer to the IP addresses above to identify the \emph{public} IPv4 addresses:
\end{questions}

\end{document}
